\documentclass[10pt,a4paper,sans]{moderncv} % Font sizes: 10, 11, or 12; paper sizes: a4paper, letterpaper, a5paper, legalpaper, executivepaper or landscape; font families: sans or roman
\usepackage[utf8]{inputenc}

\moderncvstyle{casual} % CV theme - options include: 'casual' (default), 'classic', 'oldstyle' and 'banking'
\moderncvcolor{blue} % CV color - options include: 'blue' (default), 'orange', 'green', 'red', 'purple', 'grey' and 'black'
\usepackage[scale=0.85]{geometry} % Reduce document margins (default: 0.8)
\usepackage{gensymb} %used for the degree "°" symbol
\usepackage{multicol}
\setlength{\hintscolumnwidth}{3cm} % Uncomment to change the width of the dates column
%\setlength{\makecvtitlenamewidth}{10cm} % For the 'classic' style, uncomment to adjust the width of the space allocated to your nam


\usepackage[% https://tex.stackexchange.com/questions/299915/how-to-add-a-bibliography-to-moderncv
  backend=biber,
  style=ieee,
  defernumbers=true,
  sorting=ydnt,         % Sort by year (descending), name, title
% sorting=none,         % Do not sort at all. All entries are processed in citation order.
]{biblatex}
\addbibresource{biblio.bib}

\makeatletter
\newcommand{\citesinthissection}[1]{\xdef\@totalcites{#1}}
% http://tex.stackexchange.com/q/66829/5764
\newcounter{numbibentries}
\renewbibmacro*{finentry}{\stepcounter{numbibentries}\finentry}
% http://tex.stackexchange.com/q/123805/5764
\defbibenvironment{bibliography}
  {\list
     {\printtext[labelnumberwidth]{% label format from numeric.bbx
        \printfield{labelprefix}%
        \number\numexpr\@totalcites-\abx@field@labelnumber+1\relax}}
     {\setlength{\topsep}{0pt}% layout parameters from moderncvstyleclassic.sty
      \setlength{\labelwidth}{\hintscolumnwidth}%
      \setlength{\labelsep}{\separatorcolumnwidth}%
      \leftmargin\labelwidth%
      \advance\leftmargin\labelsep}%
      \sloppy\clubpenalty4000\widowpenalty4000}
  {\endlist}
  {\item}
\makeatother

%\nopagenumbers
\usepackage{lastpage}
\nopagenumbers%
%\AtEndPreamble{%
%    \AtBeginDocument{%
%        \newlength{\pagenumberwidth} \settowidth{\pagenumberwidth}{\color{color2}\pagenumberfont\thepage~of~\pageref{LastPage}}%
%            \fancypagestyle{plain}{\fancyfoot[r]{\parbox[b]{\pagenumberwidth}{%
%    \color{color2}\pagenumberfont\thepage~of~\pageref{LastPage}
%}}}
%        \pagestyle{plain}}}

\rfoot{\color{color2}\pagenumberfont\thepage~of~\pageref{LastPage}}

% Privacy (D.Lgs. 33/2013)
\newif\ifPersonalData{}
\PersonalDatatrue{}
% \PersonalDatafalse{} % Uncomment to hide personal data

\ifPersonalData%
    \newcommand\pguard[1]{#1}
\else
    \newcommand\pguard[1]{}
\fi


\firstname{Enrico} % Your first name
\familyname{Saccon} % Your last name

% All information in this block is optional, comment out any lines you don't need

\ifPersonalData%
    %\address{Via dei Giardini, 66}{38122 Trento (TN)}{Italy}
    \mobile{+39~(339)~8242401}
    \email{enricosaccon96@gmail.com}
\fi
%\homepage{saccon.org}
\social[github]{icosac}
%\extrainfo{}
\photo[60pt][0.2pt]{../../Documenti/Fototessera-2021-10-29.png} % The first bracket is the picture height, the second is the thickness of the frame around the picture (0pt for no frame)
%\quote{"A witty and playful quotation" - John Smith}

\newcommand\Colorhref[3][black]{\href{#2}{\color{#1}#3}}
%----------------------------------------------------------------------------------------
\begin{document}
\makecvtitle% Print the CV title
%\pagestyle{empty}


    %\cvitem{\large\color{color1}RESEARCH~INTERESTS}{%
     \vspace{-0.5cm}
     {\large\color{color1}RESEARCH~INTERESTS
       \hspace*{\fill} \color{color2}
        \textsc{robotics applications, fleet management, parallel computing}
     }

\section{Education}
  \cventry{Nov 22 -- Current}{PhD in Computer Science}{University of Trento}{Italy}{}{%
    \textbf{Topic}: Multi-Agent Path Finding (MAPF), AI, Industrial Robotics. \newline
    \emph{Goal}: Develop a holistic system that through Large Language Models and logic programming is able to plan and verify schedules for industrial robots and to execute them.  
  }
  \cventry{Oct 18 -- Jul 22}{Master Degree in Computer Science}{University of
  Trento}{Italy}{Final mark: 109}{%
    \textbf{Thesis title}: ``Comparison of Multi-Agent Path Finding Algorithms for an Industrial
    Scenario`` \newline
    \textbf{Thesis argument}: managing a fleet of AGVs in a human populated 
    environment. \newline
    \textbf{Topics}: AGV control, robotics principles, path and goal planning, 
    fleet control. \newline
    Other acquired knowledge:
    \begin{itemize}
      \item Machine learning and deep learning;
      \item Real time operating systems; 
      \item Protocols and middleware for the IoT.
    \end{itemize}
  }
  \cventry{Sep 15 -- Oct 18}{Bachelor Degree in Computer Science}{University of Trento}{Italy}{}{%
      ``Implementation of GPU algorithms for robot path planning.''\newline
      \textbf{Topics}: CUDA GPU programming, robot motion planning, comfort 
      control. 
  }

\section{Fellowships}
  \cventry{Sep 22 -- Oct 22}{Research Fellowship -- ``Predoc``}{University of Trento}{Italy}{}{%
    \emph{Topics}: Multi-Agent Path Finding, fleet management\\
    \emph{Goal}: Creation of a framework encompassing different MAPF algorithms for testing and scalability analysis
  }

\section{Work Experience}
  \cventry{Sept 19 -- Dec 19}{Computer Scientist}{CreateNet -- FBK}{Italy}{}{%
    Work on cutting-edge technologies for control and optimization of
    agricultural irrigation in a large deployed system. \newline
    \emph{Topics}: C programming language, LoRaWAN infrastructure, electronic 
    sensor and actuators.
  }
  \cventry{Jan 19 -- Jul 19}{High School Teacher}{ITT Buonarroti-Pozzo}{Italy}{}{%
    Responsible for passing the interest in computer science to the next
    generation. \newline 
    1$^{st}$ year: mainly problem solving skills; \newline
    2$^{nd}$ year: basics of programming with C.
  }

\section{Research Experience}
  \cventry{Dec 20 -- May 21}{Student}{University of Trento}{Italy}{}{%
    Topics:
    \begin{itemize}
      \item Research on \textbf{Dubins} curves for optimal control of vehicles;
      \item Implementation on \textbf{GPU} of dynamic programming for 
        multi-point Dubinses;
      \item \textbf{Energetic analysis} of different solutions from embedded 
        systems to server based ones.
    \end{itemize}
  }

  \cventry{Jul 18 -- Oct 18}{Student}{University of Trento}{Italy}{}{%
    Topics:
    \begin{itemize}
      \item Implementation on GPU of \textbf{path planning algorithms} for 
        robotics applications;
      \item Parallel computing of \textbf{clothoids} using CUDA. 
    \end{itemize}
  }

%%%% PUBLICATIONS %%%%
\nocite{MAOFExtAbstract,ComparingMAPF,GPUsGameChanger,IterativeApproach}
\citesinthissection{4}
\printbibliography[title={Publications}]


\section{Public Speaking}
  \cventry{Jul 23}{Speaker}{Prague}{Czech Republic}{}{
    \textbf{16th International Symposium on Combinatorial Search (SoCS 2023)}\newline
    Presented the extended abstract for the Doctoral Consortium: "Multi-Agent Open 
    Framework: Developing a Holistic System to Solve MAPF"
  }
  \cventry{Nov 22}{Speaker}{Udine}{Italy}{}{
    \textbf{21st International Conference of the Italian Association for Artificial
    Intelligence (AIxIA 2022)} \newline
    Presented the conference paper: "Comparing Multi-Agent Path Finding Algorithms in a 
    Real Industrial Scenario"
  }
  \cventry{Jul 21}{Speaker}{Madrid (virtual)}{Spain}{}{
    \textbf{IEEE COMPSAC 2021 Intelligent and Resilient Computing for a Collaborative
    World}\newline
    Presented the conference paper: "Robot Motion Planning: can GPUs be a Game Changer?"
  }


\section{Skills}
  \cvitem{Programming Languages}{%
      C, C++, Python, Matlab, R, Latex, Java, Bash, JavaScript, PolyML
  }
  \cvitem{Technologies}{%
    Git, CUDA, Machine/Deep Learning: PyTorch, Tensorflow, Iot: Contiki-NG,
    Django, NodeJS
  }
  \cvitem{Sys Admin}{Linux}

\section{Languages}
    \cvitem{Italian}{Mother tongue}
    \cvitem{English}{Full professional knowledge and B2 certified}

\section{Communication and Interpersonal Skills}
  \cvitem{}{
    Good teamworking skills learned through various projects assigned during
    university courses and partecipations in Hackathon events (Hackathon Italia
    2017, Hackathon FBK 2016, Hackathon Google 2018) and Google Hashcode (in
    2019, 2020, 2021, 2022). \newline
    Took part in the CoderDojo project for a few months during 2017. The project
    aims to teach younger people the beauty of coding. This was another
    occasion to practice teamworking and to test myself.
  }\vspace{-0.4cm}


\vspace*{\fill}


    \noindent\makebox[\linewidth]{\rule{.9\paperwidth}{0.4pt}}
    {\center\footnotesize
        % From template:
        % In compliance with the decree no. 196 dated 30/06/2003, I authorize the use of my personal details.\newline
        % Le informazioni qui contenute vengono rese ai sensi e per gli effetti degli artt. 46 e 47 del DPR 445/2000.
        I hereby grant permission for the treatment of my personal data (in compliance with Italian law 196 – 06/30/2003 and art. 13 GDPR (2016/679) for all purposes related to the selection procedure.
        % From UniTN:
        % Le informazioni qui contenute vengono rese ai sensi e per gli effetti degli artt. 46 e 47 del DPR 445/2000.
        % Acconsento alla pubblicazione del mio CV in ottemperanza alle disposizioni di legge dettate in materia di trasparenza (D.Lgs. 33/2013).
    }

\lfoot{\today}
\cfoot{Enrico Saccon}
\thispagestyle{fancy}
\end{document}
